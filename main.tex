\documentclass[10pt, twoside]{article}
\usepackage{color}
\usepackage{wrapfig}
\definecolor{light-gray}{gray}{0.95}
\usepackage{tcolorbox}
\usepackage{geometry}
\geometry{letterpaper, portrait, margin=0.5in}
\usepackage[utf8]{inputenc}
\usepackage{helvet}
\usepackage{url}
\usepackage[hidelinks]{hyperref}
\usepackage[round]{natbib}


% -------- SECTION FORMATTING ----------
\usepackage{titlesec}

\setcounter{secnumdepth}{0}

% Section
\titleformat{\section}
  {\large\bfseries}
  {\thesection.}
  {0.5em}
  {}

% Subsection
\titleformat{\subsection}
  {\normalsize\bfseries}
  {\thesubsection.}
  {0.5em}
  {}

% Subsubsection
\titleformat{\subsubsection}
  {\normalsize\bfseries\itshape}
  {\thesubsubsection.}
  {0.5em}
  {}

% Optional: spacing
\titlespacing*{\section}{0pt}{3mm}{2mm}
\titlespacing*{\subsection}{0pt}{2mm}{1mm}
\titlespacing*{\subsubsection}{0pt}{1mm}{1mm}


% -------- FIGURE CAPTIONS ----------
\usepackage{caption}

\usepackage{caption}

\DeclareCaptionLabelSeparator{dash}{\textemdash}

\captionsetup[figure]{
  name=Fig.,
  labelfont=bf,
  textfont=bf,
  font=footnotesize,
  labelsep=dash,
  justification=raggedright,
  singlelinecheck=false
}

% ---- Tables (label normal, text normal) ----
\captionsetup[table]{
  name=Table,
  labelfont=normalfont,
  textfont=normalfont,
  font=footnotesize,
  labelsep=dash,
  justification=raggedright,
  singlelinecheck=false
}

\begin{document}
\begin{flushleft}
{\normalsize Enter: Journal or Event Name}
\end{flushleft}

\begin{flushleft}
{\normalsize Enter: Paper Number (SPE-XXXXXX) or Manuscript ID}
\end{flushleft}
\vspace{5mm}

\begin{center}
    {\Huge Title}
\end{center}
\vspace{5mm}

\begin{flushleft}
\textbf{Enter: A. B. Author$^{1*}$,} \textbf{B. C. Author$^{2}$}, \textbf{D. E. Author$^{2}$,} \rm{and} \  \textbf{F. G. Author$^{3}$}
\end{flushleft}
\vspace{3mm}


\begin{flushleft}
\normalsize{$^{1}$\rm{Department, University, City, State, Country}}
\end{flushleft}
\begin{flushleft}
\normalsize{$^{2}$\rm{Company, City, State, Country}}
\end{flushleft}
\begin{flushleft}
\normalsize{$^{3}$\rm{University/Company, City, State, Country\ (now\ with\  XYZ)}}
\end{flushleft}
\vspace{3mm}

\begin{flushleft}
\normalsize*Corresponding author; email: \url{author@email.com} \textbf{(Journals only; delete if submitting to a conference.)}
\end{flushleft}
\vspace{5mm}

\section{First-Level Heading new}

\begin{flushleft}
\large{\textbf{First-Level Heading}}\\
\end{flushleft}

\setlength{\parindent}{0pt}
\justifying{\normalsize{Body text 1 paragraph.}}
\vspace{3mm}

\setlength{\parindent}{20pt}
\justifying{\normalsize{Body text 2 paragraph.}}
\vspace{5mm}

\subsection{Second-Level Heading new}

\begin{flushleft}
\normaltext{{\textbf{Second-Level Heading.}}}
\end{flushleft}

\subsubsection{Third-Level Heading new}

\setlength{\parindent}{20pt}
\normaltext{{\textbf{\textit{Third-Level Heading.}}}}\\

\setlength{\parindent}{20pt}
\normaltext{{\textit{Fourth-Level Heading.}}}\\
\vspace{5mm}




\setlength{\parindent}{0pt}
\justifying
{\footnotesize{\textbf{Fig. 1—Figure captions should begin with an overall descriptive statement of the figure followed by additional supporting text. Captions should be placed immediately after each figure. Figure parts are indicated with lower-case letters: (a) Part 1; (b) Part 2; (c) Part 3.}}}
\vspace{5mm}

\begin{figure}[b!]
    \centering
    Here is my figure.
    \caption{Figure caption.}
    \label{fig:placeholder}
\end{figure}



\newpage

\begin{table}[t!]
    \centering
    Here is my table.
    \caption{Table caption.}
    \label{tab:placeholder}
\end{table}

\setlength{\parindent}{0em}
{\footnotesize{Table 1—Table captions should begin with a short description of the table. Format tables using the Microsoft Word table commands and structures. Do not create tables using spaces or tab characters. Large tables presenting rich data should be presented as separate Excel or .csv files, not as part of the main text.}}
\vspace{5mm}

\section{Test citation}

Developing \citet{alyaev2026template}, we use an improved \LaTeX template \citep{alyaev2026template}.


\section{Acknowledgemets}
The manuscript is prepared in Latex using the template from Sergey Alyaev:\\ 
\url{https://github.com/alin256/spe-2023-latex-template-done-right}.

\bibliographystyle{kluwer}
\bibliography{lib}

\end{document}


