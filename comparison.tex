
\begin{flushleft}
\large{\textbf{First-Level Heading}}\\
\end{flushleft}


\setlength{\parindent}{0pt}
\justifying{\normalsize{Body text 1 paragraph.}}
\vspace{3mm}

\setlength{\parindent}{20pt}
\justifying{\normalsize{Body text 2 paragraph.}}
\vspace{5mm}

\begin{flushleft}
\normaltext{{\textbf{Second-Level Heading.}}}
\end{flushleft}

\setlength{\parindent}{20pt}
\normaltext{{\textbf{\textit{Third-Level Heading.}}}}\\

\setlength{\parindent}{20pt}
\normaltext{{\textit{Fourth-Level Heading.}}}\\
\vspace{5mm}




\setlength{\parindent}{0pt}
\justifying
{\footnotesize{\textbf{Fig. 1—Figure captions should begin with an overall descriptive statement of the figure followed by additional supporting text. Captions should be placed immediately after each figure. Figure parts are indicated with lower-case letters: (a) Part 1; (b) Part 2; (c) Part 3.}}}
\vspace{5mm}






\setlength{\parindent}{0em}
{\footnotesize{Table 1—Table captions should begin with a short description of the table. Format tables using the Microsoft Word table commands and structures. Do not create tables using spaces or tab characters. Large tables presenting rich data should be presented as separate Excel or .csv files, not as part of the main text.}}
\vspace{5mm}
